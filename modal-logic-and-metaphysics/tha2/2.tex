% Ontology and Metaontology Ch. 11 Possible Worlds
\section*{QUESTION 2}
The difference lies in their ontological commitments. Genuine modal realism commits to a plurality of concrete worlds, each as real as our own, but inaccessible to us. Modal actualism, on the other hand, maintains a more conservative ontology, positing that only the actual world is concrete, and possible worlds are merely abstract ways of talking about how certain things could have been different.

Under the view of genuine modal realism, possible worlds are concrete entities, complete with their own space, time, and causal relations. They are real in the same sense that our world is real, albeit completely separate from our own. For Lewis, this means that when we talk about how things could have been, we are referring to the way things are in some other concrete world. This approach provides a straightforward way to analyse modal claims by referring to what is true in these concrete worlds. However, this ontological generosity postulates an infinite number of fully-fledged, concrete universes, which is seen as a costly commitment.

In contrast, modal actualism avoids such ontological extravagance by claiming that all possible worlds are abstract rather than concrete. These abstract entities can take various forms, such as maximal consistent sets of propositions, sentences, states of affairs, or properties. They are not real worlds but representations of how the world could be. On this view, when we talk about possibilities, we are not referring to other concrete worlds but rather to abstract representations of different ways the world might have been. This approach allows for a discussion of modal claims without committing to the existence of an infinite number of fully-fledged, concrete universes. However, modal actualism struggles to define the notion of possibility without committing to primitive modality. Modal actualism also faces the issue of cross-world identity. While genuine modal realism can straightforwardly claim that there are distinct counterparts in different concrete worlds, modal actualism must navigate the more abstract terrain of identifying entities across different possible representations. Moreover, the abstract nature of possible worlds in modal actualism may not provide the same richness and depth of explanatory power as the concrete worlds posited by genuine modal realism.
