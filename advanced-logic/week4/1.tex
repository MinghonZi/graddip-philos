\begin{itemize}
\item 
$f \colon \omega \to \omega$ given by $f(x) = x + 2$ is injective because $(\forall x,y \in \omega)(f(x) = f(y) \to x = y)$. It is not surjective because $\mathrm{range}(f) \not\equal \omega$ ($0$ and $1$ are not in $\mathrm{range}(f)$).

\item 
$\dfrac{0}{2} = 0, \dfrac{2}{2} = 1, \dfrac{4}{2} = 2, \dfrac{6}{2} = 3, \dots$

$\dfrac{1-1}{2} = 0, \dfrac{3-1}{2} = 1, \dfrac{5-1}{2} = 2, \dfrac{7-1}{2} = 3, \dots$

The map is surjective because $\mathrm{range}(g) \equal \omega$. It is not injective because for each $y \in \omega$ there are two $x \in \omega$ such that $g(x) = y$.

\item
$x \mapsto x + 1$ when $x$ is an even natural number yields all odd natural numbers. $x \mapsto x - 1$ when $x$ is an odd natural number yields all even natural numbers. The function is injective because it never maps two different arguments to the same value. It is also surjective because it has every element of the codomain as a value, viz., all odd natural numbers $+$ all even natural numbers $= \omega$. Therefore, the function is bijective.

\item 
It is not injective because it maps different arguments to the same value. It is not surjective because it does not have every element of the codomain as a value.

\end{itemize}