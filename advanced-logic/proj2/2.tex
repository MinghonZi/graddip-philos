% Discussed in MSE4125051: https://math.stackexchange.com/questions/4125051/problem-17-2-from-computability-and-logic-by-boolos-et-all
\section*{QUESTION 1B: RECURSIVE SETS IN ARITHMETIC}

We wish to show that there is no recursive set $S$ such that both $P^+ \subset S$ and $P^- \cap S = \emptyset$.

\begin{proof}
Suppose $S$ is recursive, then there is a formula $F(x)$ that defines $S$ in $Q$:
\begin{align}
   n \in S &\iff Q \vdash F(\underline{n}) \label{eq:1} \\
   n \not\in S &\iff Q \vdash \neg F(\underline{n}) \label{eq:2}
\end{align}
where $\underline{n}$ is the numeral of $n$. By the diagonalisation lemma, we can find a sentence $G$ such that
\begin{equation}
   T \vdash G \leftrightarrow \neg F(\ulcorner G \urcorner) \label{eq:3}
\end{equation}
Suppose $G$ is a theorem of $T$, then $\ulcorner G \urcorner \in P^+$ by the definition of $P^+$. Since $P^+ \subset S$ according to the definition of $S$, we have $\ulcorner G \urcorner \in S$. The left-to-right direction of (\ref{eq:1}) gives $Q \vdash F(\ulcorner G \urcorner)$. Since $T$ extends $Q$, we can weaken the last conclusion to $T \vdash F(\ulcorner G \urcorner)$. Consequently, the left-to-right direction of the contraposition of (\ref{eq:3})
\begin{equation*}
   T \vdash F(\ulcorner G \urcorner) \leftrightarrow \neg G
\end{equation*}
tells us $T \vdash \neg G$. This implies $\ulcorner G \urcorner \in P^-$ by the definition of $P^-$. According to the definition of $S$, $P^- \cap S = \emptyset$, so $\ulcorner G \urcorner \not\in S$. But this contradicts $\ulcorner G \urcorner \in S$.

Suppose $G$ is not a theorem of $T$, then by the definition of $P^-$, $\ulcorner G \urcorner \in P^-$ . Since $P^- \cap S = \emptyset$ according to the definition of $S$, we have $\ulcorner G \urcorner \not\in S$. The left-to-right direction of (\ref{eq:2}) gives $Q \vdash \neg F(\ulcorner G \urcorner)$. Since $T$ extends $Q$, we can weaken the last conclusion to $T \vdash \neg F(\ulcorner G \urcorner)$. Consequently, by the right-to-left direction of (\ref{eq:3}), $T \vdash G$. This implies $\ulcorner G \urcorner \in P^+$ by the definition of $P^+$. According to the definition of $S$, $P^+ \subset S$, so $\ulcorner G \urcorner \in S$. But this contradicts $\ulcorner G \urcorner \not\in S$.

Since both possibilities lead to a contradiction, we conclude that $S$ cannot be recursive.
\end{proof}