We first use the properties of maximal consistent sets \textcolor{Maroon}{(vii)} and the properties of witnessed sets \textcolor{Maroon}{(ii)} to prove that a witnessed maximal consistent set is a theory \textcolor{Maroon}{(iii)}. Next, we use this fact to extend the assumption $X \not\vdash A$ to $X^* \not\vdash A$ where $X^*$ is a witnessed maximal consistent set that contains the arbitrary set $X$ \textcolor{Maroon}{(i)}. Then, we can construct a counterexample over $X^* \Yright A$ which makes each member of $X^*$ true and $A$ false, i.e. $X^* \not\vDash A$ \textcolor{Maroon}{(v)}. Since $X \subseteq X^*$, this counterexample also makes each member of $X$ true, so $X \not\vDash A$. By discharging the assumption $X \not\vdash A$ in (i), we conclude if $X \not\vdash A$ then $X \not\vDash A$ \textcolor{Maroon}{(vi)}. Finally, the completeness theorem is the contrapositive of the last step's conclusion \textcolor{Maroon}{(iv)}.
