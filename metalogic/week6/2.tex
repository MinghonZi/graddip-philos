\begin{enumerate}[label=(\roman*)]
\item
$\forall x \forall y (Pxy \to Qxy)$ means $P \subseteq Q$.

$\forall x \forall y (Pxy \to Qxy) \land \forall x \forall y \forall z ((Pxy \land Qyz) \to Qxz)$, i.e., $Repeat(P, Q)$, means $Q$ is a transitive relation contains $P$. \textcolor{red}{Don't know how to explain, so might be wrong.}

By observing the tables, $S$ is a transitive relation \textcolor{red}{(Why?)}, and it does contain $R$. Therefore, $Repeat(R, S)$ holds in this model.

\item
$Repeat(P, P^*)$: $P^*$ is a transitive relation contains $P$.

$\forall Q (Repeat(P, Q) \to \forall x \forall y (P^*xy \to Qxy))$: if $Q$ is any transitive relation containing $P$, then $P^* \subseteq Q$.

The conjunction of them means that $P^*$ is the smallest transitive relation containing $P$, namely, $P^*$ is the transitive closure of $P$.

The interpretation of $R^*$ so that it is the transitive closure of $R$ is
\begin{center}
\begin{tblr}{
  colspec={c|cccccc},
}
$I(R^*)$ & $\alpha$ & $\beta$ & $\gamma$ & $\delta$ & $\epsilon$ & $\zeta$\\
\hline[solid]
$\alpha$   & $0$ & $0$ & $1$ & $1$ & $1$ & $1$ \\
$\beta$    & $0$ & $0$ & $1$ & $1$ & $1$ & $1$ \\
$\gamma$   & $0$ & $0$ & $0$ & $1$ & $1$ & $1$ \\
$\delta$   & $0$ & $0$ & $0$ & $0$ & $1$ & $1$ \\
$\epsilon$ & $0$ & $0$ & $0$ & $0$ & $0$ & $0$ \\
$\zeta$    & $0$ & $0$ & $0$ & $0$ & $0$ & $0$
\end{tblr}
\end{center}
\textcolor{red}{Why?}

\item
Every relation has a transitive closure. \textcolor{red}{Don't know how to prove using the second-order logic.}

\end{enumerate}