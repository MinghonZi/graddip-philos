\begin{enumerate}[label=\alph*)]
\item[b)]

$M_1$ and $M_2$ are counterexamples to this argument.

In $M_1$, $I_1(\forall x \exists y Rxy, \alpha) = 1$, since
\begin{itemize}[topsep=-\parskip, noitemsep, leftmargin=*]
\item $I_1(\exists y Rxy, \alpha[x : k]) = 1$, since $I(Rxy, \alpha[x : k, y : m]) = 1$,
\item $I_1(\exists y Rxy, \alpha[x : l]) = 1$, since $I(Rxy, \alpha[x : l, y : k]) = 1$,
\item $I_1(\exists y Rxy, \alpha[x : m]) = 1$, since $I(Rxy, \alpha[x : m, y : l]) = 1$.
\end{itemize}
$I_1(\forall x Rxx, \alpha) = 0$, since
\vspace{-\parskip}
\begin{itemize}[noitemsep, leftmargin=*]
\item $I_1(Rxx, \alpha[x : k]) = 0$,
\item $I_1(Rxx, \alpha[x : l]) = 0$,
\item $I_1(Rxx, \alpha[x : m]) = 0$.
\end{itemize}

In $M_2$, $I_2(\forall x \exists y Rxy, \alpha) = 1$, since
\begin{itemize}[topsep=-\parskip, noitemsep, leftmargin=*]
\item $I_1(\exists y Rxy, \alpha[x : k]) = 1$, since $I(Rxy, \alpha[x : k, y : l]) = 1$,
\item $I_1(\exists y Rxy, \alpha[x : l]) = 1$, since $I(Rxy, \alpha[x : l, y : k]) = 1$,
\item $I_1(\exists y Rxy, \alpha[x : m]) = 1$, since $I(Rxy, \alpha[x : m, y : k]) = 1$.
\end{itemize}
$I_1(\forall x Rxx, \alpha) = 0$, since
\vspace{-\parskip}
\begin{itemize}[noitemsep, leftmargin=*]
\item $I_1(Rxx, \alpha[x : k]) = 0$,
\item $I_1(Rxx, \alpha[x : l]) = 0$,
\item $I_1(Rxx, \alpha[x : m]) = 0$.
\end{itemize}

$M_3$ is not a counterexample to this argument because $I_3(\forall x \exists y Rxy, \alpha) = 0$:
\vspace{-\parskip}
\begin{itemize}[noitemsep, leftmargin=*]
\item $I_3(Rxy, \alpha[x : m, y : k]) = 0$,
\item $I_3(Rxy, \alpha[x : m, y : l]) = 0$,
\item $I_3(Rxy, \alpha[x : m, y : m]) = 0$.
\end{itemize}

\end{enumerate}